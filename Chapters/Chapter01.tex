%************************************************
\chapter{Introduction}\label{ch:introduction}
%************************************************
Magnetic Resonance Imaging (MRI) is a fundamental tool in radiology and biomedical research, offering capabilities for both anatomical and functional imaging. MRI excels in providing superior soft-tissue contrast compared to CT scans and delivers higher resolution images than ultrasound, all while being a generally safe method with infrequent occurrences of patient harm \cite{incidentRates} and no exposure to ionizing radiation.\\

MRI operates based on the Nuclear Magnetic Resonance (NMR) phenomenon to create changes in magnetization that are detectable by its receiver systems. It achieves spatial localization by stimulating nuclear spins within an external magnetic field. By modifying the imaging sequences, MRI can manipulate spin systems to produce various contrasts, such as relaxation, proton density, diffusion, and phase contrasts. Beyond producing diagnostic-quality images, MRI can also be adapted for quantitative analysis, which has spurred developments in creating maps of tissue physical properties for quantitative clinical interpretations.\cite{quantitativeMri}\\

Globally, over 25,000 MRI scanners are in use, supporting a wide range of diagnostic and therapeutic applications. In neuroimaging, MRI is crucial for distinguishing between gray and white matter, aiding in the diagnosis of conditions like dementia, Alzheimer’s disease, demyelinating diseases, epilepsy, and anomalies in the brain and spinal cord. It also facilitates diffusion and functional imaging techniques that can map neuronal tracts and blood flow. Cardiovascular uses of MRI include examining the structure and function of the heart and assessing vascular diseases. In musculoskeletal imaging, MRI is used for evaluating joints, spine, soft tissue tumors, and muscle disorders. Additionally, MRI is employed in abdominal assessments for the liver, gastrointestinal tract, breasts, and prostate, particularly useful in detecting cysts, tumors, and other abnormalities. Functional imaging of metabolites through spectroscopy is also a capability of MRI.
\cite{wikipediaMri}