%************************************************
\chapter{Introduction}\label{ch:introduction}
%************************************************
Magnetic Resonance Imaging (MRI) is a fundamental tool in radiology and biomedical research, offering capabilities for both anatomical and functional imaging. MRI excels in providing superior soft-tissue contrast compared to CT scans and delivers higher resolution images than ultrasound, all while being a generally safe method with infrequent occurrences of patient harm \cite{incidentRates} and no exposure to ionizing radiation.\\

MRI operates based on the Nuclear Magnetic Resonance (NMR) phenomenon to create changes in magnetization that are detectable by its receiver systems. It achieves spatial localization by stimulating nuclear spins within an external magnetic field. By modifying the imaging sequences, MRI can manipulate spin systems to produce various contrasts, such as relaxation, proton density, diffusion, and phase contrasts. Beyond producing diagnostic-quality images, MRI can also be adapted for quantitative analysis, which has spurred developments in creating maps of tissue physical properties for quantitative clinical interpretations.\cite{quantitativeMri}\\

Globally, over 25,000 MRI scanners are in use, supporting a wide range of diagnostic and therapeutic applications. In neuroimaging, MRI is crucial for distinguishing between gray and white matter, aiding in the diagnosis of conditions like dementia, Alzheimer’s disease, demyelinating diseases, epilepsy, and anomalies in the brain and spinal cord. It also facilitates diffusion and functional imaging techniques that can map neuronal tracts and blood flow. Cardiovascular uses of MRI include examining the structure and function of the heart and assessing vascular diseases. In musculoskeletal imaging, MRI is used for evaluating joints, spine, soft tissue tumors, and muscle disorders. Additionally, MRI is employed in abdominal assessments for the liver, gastrointestinal tract, breasts, and prostate, particularly useful in detecting cysts, tumors, and other abnormalities. Functional imaging of metabolites through spectroscopy is also a capability of MRI.
\cite{wikipediaMri}\\

In diagnostic imaging, fat presents a crucial MR signal component that often needs to be suppressed or differentiated from water signals. This manipulation is vital for accurate quantification of MR properties in tissues characterized by mixed chemical shifts. Particularly, imaging of short-\(T_2\) tissues requires significant suppression of long-\(T_2\) signals from both water and fat to enhance image contrast and clarity. The technology of water-fat (WF) imaging has evolved into state-of-the-art imaging methodology essential for differentiating between water and fat signals. These methods are particularly useful for qualitative purposes, such as suppressing the fat signal, or they are used quantitatively to derive biomarkers of tissue fat concentration such as the Proton Density Fat Fraction (PDFF). Furthermore, water-fat MRI is particularly useful for assessing diseases linked to metabolic disorders, such as obesity, metabolic syndrome, and type-2 diabetes \cite{water_fat_1, water_fat_2}. The ability to effectively manage fat signals in MRI is useful for improving the diagnostic accuracy of MRI scans and they not only provide a clearer visualization of anatomical structures but also facilitate a better understanding of tissue compositions, which is crucial for accurate diagnosis and treatment planning. \cite{chapter}\\

This method utilizes chemical shift encoding (CSE) techniques, also known as Dixon imaging, capitalizing on the chemical shift differences between water and fat MR signals to differentiate them and create the two respective distinct images. The water-fat separation process involves fitting the acquired signal to a physical model, typically necessitating multiple image captures at different echo times. This requirement extends the scan duration, and coupled with the inherently slow MRI acquisition speed, increases the susceptibility of the technique to motion artifacts from patient or physiological movement. Techniques such as respiratory triggering or breath-hold are employed to minimize these artifacts, though they may elevate patient discomfort. Therefore, there is a pressing need to develop methods to expedite the MRI scanning process.\\

The challenge of prolonged scanning times in MRI is well-recognized, with a significant body of research focused on accelerating the acquisition process through fast sequences and parallel imaging techniques. Under this context multi-echo and single-echo ultra short echo time (UTE) sequences have been developed and studied to accelerate this scanning time \cite{ute}. Specifically single-echo/single-point UTE Dixon techniques have been developed for fast acquisition with promising separation results since the short echo times make the \(R_2^*\) relaxation negligible; but it amplifies the relative effects of \(B_0 \: \& \: B_1\) field inhomogeneities affecting the quality of the water-fat separation, since the signal is not strong. Moreover, by using only one complex image the time dependent effects of the signal cannot be inferred, translating into a more complex that requires background phase corrections for a reliable WF separation.\\

Because of these pitfalls some techniques have been developed for single-point UTE Dixon like Kronthaler S., et. al \cite{kronthaler}, which use an iterative optimization method to solve the water-fat separation problem, trying to capture the background phase to remove it from the signal and continue with the classic separation algorithm. Although the algorithm tries to regularize the image by rewarding smoothness of the fitted phase, the problem is still ill posed and without proper initialization the stability of the algorithm may not deliver its full potential.\\

The purpose of this study is to analyze different modeling approaches of background phase contributions and various initialization techniques for the algorithm in Kronthaler S., et. al.; comparing it qualitatively with the non-altered model and studying its effects on the resulting water-fat separation quality and stability of the algorithm.\\

\subsection{Thesis Structure}
The thesis is organized into the following chapters:
\begin{itemize}
\item \textbf{Chapter 1}: The current chapter presents a brief introduction, motivation and purpose of the thesis.
\item \textbf{Chapter 2}: Resumes the general theory of MRI, including physics theory, signal encoding, k-space and echo sequences.
\item \textbf{Chapter 3}: Introduces the UTE sequences and the water fat separation of MRI signals.
\item \textbf{Chapter 4}: Analyses the different background phase contributions and its effects in the resulting MRI acquisition.
\item \textbf{Chapter 5}: Introduces the iterative optimization method used for water-fat separation and analyses the impact of different initialization strategies on its resulting water and fat images.
\item \textbf{Chapter 6}: Summarizes the conclusions drawn from various results from the previous
chapters.
\end{itemize}
